\documentclass[a4paper, 12pt]{book}
%\documentclass[a4paper, 12pt, draft]{book}  Nalogo preverite tudi z opcijo draft, ki vam bo pokazala, katere vrstice so predolge!



\usepackage[utf8x]{inputenc}   % omogoča uporabo slovenskih črk kodiranih v formatu UTF-8
\usepackage[slovene,english]{babel}    % naloži, med drugim, slovenske delilne vzorce
\usepackage[pdftex]{graphicx}  % omogoča vlaganje slik različnih formatov
\usepackage{fancyhdr}          % poskrbi, na primer, za glave strani
\usepackage{amssymb}           % dodatni simboli
\usepackage{amsmath}           % eqref, npr.
\usepackage[hyphens]{url}  % dodal Solina
\usepackage{comment}       % dodal Solina

\usepackage[pdftex, colorlinks=true,
						citecolor=black, filecolor=black, 
						linkcolor=black, urlcolor=black,
						pagebackref=false, 
						pdfproducer={LaTeX}, pdfcreator={LaTeX}, hidelinks]{hyperref}

\usepackage{color}       % dodal Solina
\usepackage{soul}       % dodal Solina

%%%%%%%%%%%%%%%%%%%%%%%%%%%%%%%%%%%%%%%%
%	DIPLOMA INFO
%%%%%%%%%%%%%%%%%%%%%%%%%%%%%%%%%%%%%%%%
\newcommand{\ttitle}{Mobilna aplikacija Asana Master}
\newcommand{\ttitleEn}{Mobile app Asana Master}
\newcommand{\tsubject}{\ttitle}
\newcommand{\tsubjectEn}{\ttitleEn}
\newcommand{\tauthor}{Klementina Garbajs}
\newcommand{\tkeywords}{Joga, Mobilna aplikacija}
\newcommand{\tkeywordsEn}{Yoga, Mobile app}


%%%%%%%%%%%%%%%%%%%%%%%%%%%%%%%%%%%%%%%%
%	HYPERREF SETUP
%%%%%%%%%%%%%%%%%%%%%%%%%%%%%%%%%%%%%%%%
\hypersetup{pdftitle={\ttitle}}
\hypersetup{pdfsubject=\ttitleEn}
\hypersetup{pdfauthor={\tauthor, kg4597@student.uni-lj.si}}
\hypersetup{pdfkeywords=\tkeywordsEn}

%%%%%%%%%%%%%%%%%%%%%%%%%%%%%%%%%%%%%%%%
% postavitev strani
%%%%%%%%%%%%%%%%%%%%%%%%%%%%%%%%%%%%%%%%  

\addtolength{\marginparwidth}{-20pt} % robovi za tisk
\addtolength{\oddsidemargin}{40pt}
\addtolength{\evensidemargin}{-40pt}

\renewcommand{\baselinestretch}{1.3} % ustrezen razmik med vrsticami
\setlength{\headheight}{15pt}        % potreben prostor na vrhu
\renewcommand{\chaptermark}[1]%
{\markboth{\MakeUppercase{\thechapter.\ #1}}{}} \renewcommand{\sectionmark}[1]%
{\markright{\MakeUppercase{\thesection.\ #1}}} \renewcommand{\headrulewidth}{0.5pt} \renewcommand{\footrulewidth}{0pt}
\fancyhf{}
\fancyhead[LE,RO]{\sl \thepage}
\fancyhead[RE]{\sc \tauthor}              % dodal Solina
\fancyhead[LO]{\sc Diplomska naloga}     % dodal Solina


\newcommand{\BibTeX}{{\sc Bib}\TeX}

%%%%%%%%%%%%%%%%%%%%%%%%%%%%%%%%%%%%%%%%
% naslovi
%%%%%%%%%%%%%%%%%%%%%%%%%%%%%%%%%%%%%%%%  


\newcommand{\autfont}{\Large}
\newcommand{\titfont}{\LARGE\bf}
\newcommand{\clearemptydoublepage}{\newpage{\pagestyle{empty}\cleardoublepage}}
\setcounter{tocdepth}{1}	      % globina kazala

%%%%%%%%%%%%%%%%%%%%%%%%%%%%%%%%%%%%%%%%
% konstrukti
%%%%%%%%%%%%%%%%%%%%%%%%%%%%%%%%%%%%%%%%  
\newtheorem{izrek}{Izrek}[chapter]
\newtheorem{trditev}{Trditev}[izrek]
\newenvironment{dokaz}{\emph{Dokaz.}\ }{\hspace{\fill}{$\Box$}}

%%%%%%%%%%%%%%%%%%%%%%%%%%%%%%%%%%%%%%%%%%%%%%%%%%%%%%%%%%%%%%%%%%%%%%%%%%%%%%%
%% PDF-A
%%%%%%%%%%%%%%%%%%%%%%%%%%%%%%%%%%%%%%%%%%%%%%%%%%%%%%%%%%%%%%%%%%%%%%%%%%%%%%%


%%%%%%%%%%%%%%%%%%%%%%%%%%%%%%%%%%%%%%%% 
% define medatata
%%%%%%%%%%%%%%%%%%%%%%%%%%%%%%%%%%%%%%%% 
\def\Title{\ttitle}
\def\Author{\tauthor, kg4597@student.uni-lj.si}
\def\Subject{\ttitleEn}
\def\Keywords{\tkeywordsEn}

%%%%%%%%%%%%%%%%%%%%%%%%%%%%%%%%%%%%%%%% 
% \convertDate converts D:20080419103507+02'00' to 2008-04-19T10:35:07+02:00
%%%%%%%%%%%%%%%%%%%%%%%%%%%%%%%%%%%%%%%% 
\def\convertDate{%
    \getYear
}

{\catcode`\D=12
 \gdef\getYear D:#1#2#3#4{\edef\xYear{#1#2#3#4}\getMonth}
}
\def\getMonth#1#2{\edef\xMonth{#1#2}\getDay}
\def\getDay#1#2{\edef\xDay{#1#2}\getHour}
\def\getHour#1#2{\edef\xHour{#1#2}\getMin}
\def\getMin#1#2{\edef\xMin{#1#2}\getSec}
\def\getSec#1#2{\edef\xSec{#1#2}\getTZh}
\def\getTZh +#1#2{\edef\xTZh{#1#2}\getTZm}
\def\getTZm '#1#2'{%
    \edef\xTZm{#1#2}%
    \edef\convDate{\xYear-\xMonth-\xDay T\xHour:\xMin:\xSec+\xTZh:\xTZm}%
}

\expandafter\convertDate\pdfcreationdate 

%%%%%%%%%%%%%%%%%%%%%%%%%%%%%%%%%%%%%%%%
% get pdftex version string
%%%%%%%%%%%%%%%%%%%%%%%%%%%%%%%%%%%%%%%% 
\newcount\countA
\countA=\pdftexversion
\advance \countA by -100
\def\pdftexVersionStr{pdfTeX-1.\the\countA.\pdftexrevision}


%%%%%%%%%%%%%%%%%%%%%%%%%%%%%%%%%%%%%%%%
% XMP data
%%%%%%%%%%%%%%%%%%%%%%%%%%%%%%%%%%%%%%%%  
\usepackage{xmpincl}
\includexmp{pdfa-1b}

%%%%%%%%%%%%%%%%%%%%%%%%%%%%%%%%%%%%%%%%
% pdfInfo
%%%%%%%%%%%%%%%%%%%%%%%%%%%%%%%%%%%%%%%%  
\pdfinfo{%
    /Title    (\ttitle)
    /Author   (\tauthor, kg4597@student.uni-lj.si)
    /Subject  (\ttitleEn)
    /Keywords (\tkeywordsEn)
    /ModDate  (\pdfcreationdate)
    /Trapped  /False
}


%%%%%%%%%%%%%%%%%%%%%%%%%%%%%%%%%%%%%%%%%%%%%%%%%%%%%%%%%%%%%%%%%%%%%%%%%%%%%%%
%%%%%%%%%%%%%%%%%%%%%%%%%%%%%%%%%%%%%%%%%%%%%%%%%%%%%%%%%%%%%%%%%%%%%%%%%%%%%%%

\begin{document}
\selectlanguage{slovene}
\frontmatter
\setcounter{page}{1} %
\renewcommand{\thepage}{}       % preprecimo težave s številkami strani v kazalu
\newcommand{\sn}[1]{"`#1"'}                    % dodal Solina (slovenski narekovaji)

%%%%%%%%%%%%%%%%%%%%%%%%%%%%%%%%%%%%%%%%
%naslovnica
 \thispagestyle{empty}%
   \begin{center}
    {\large\sc Univerza v Ljubljani\\%
      Fakulteta za računalništvo in informatiko}%
    \vskip 10em%
    {\autfont \tauthor\par}%
    {\titfont \ttitle \par}%
    {\vskip 3em \textsc{DIPLOMSKO DELO\\[5mm]         % dodal Solina za ostale študijske programe
%    VISOKOŠOLSKI STROKOVNI ŠTUDIJSKI PROGRAM\\ PRVE STOPNJE\\ RAČUNALNIŠTVO IN INFORMATIKA}\par}%
    UNIVERZITETNI  ŠTUDIJSKI PROGRAM\\ PRVE STOPNJE\\ RAČUNALNIŠTVO IN INFORMATIKA}\par}%
%    INTERDISCIPLINARNI UNIVERZITETNI\\ ŠTUDIJSKI PROGRAM PRVE STOPNJE\\ RAČUNALNIŠTVO IN MATEMATIKA}\par}%
%    INTERDISCIPLINARNI UNIVERZITETNI\\ ŠTUDIJSKI PROGRAM PRVE STOPNJE\\ UPRAVNA INFORMATIKA}\par}%
%    INTERDISCIPLINARNI UNIVERZITETNI\\ ŠTUDIJSKI PROGRAM PRVE STOPNJE\\ MULTIMEDIJA}\par}%
    \vfill\null%
    {\large \textsc{Mentor}: doc.\ dr.  Mira Trebar\par}%
    {\vskip 2em \large Ljubljana, 2021 \par}%
\end{center}
% prazna stran
%\clearemptydoublepage      % dodal Solina (izjava o licencah itd. se izpiše na hrbtni strani naslovnice)

%%%%%%%%%%%%%%%%%%%%%%%%%%%%%%%%%%%%%%%%
%copyright stran
\thispagestyle{empty}
\vspace*{8cm}

\noindent
{\sc Copyright}. 
Rezultati diplomske naloge so intelektualna lastnina avtorja in Fakultete za računalništvo in informatiko Univerze v Ljubljani.
Za objavo in koriščenje rezultatov diplomske naloge je potrebno pisno privoljenje avtorja, Fakultete za računalništvo in informatiko ter mentorja.

\begin{center}
\mbox{}\vfill
\emph{Besedilo je oblikovano z urejevalnikom besedil \LaTeX.}
\end{center}
% prazna stran
\clearemptydoublepage

%%%%%%%%%%%%%%%%%%%%%%%%%%%%%%%%%%%%%%%%
% stran 3 med uvodnimi listi
\thispagestyle{empty}
\vspace*{4cm}

\noindent
Fakulteta za računalništvo in informatiko izdaja naslednjo nalogo:
\medskip
\begin{tabbing}
\hspace{32mm}\= \hspace{6cm} \= \kill

Tematika naloge:
\end{tabbing}
Besedilo teme diplomskega dela študent prepiše iz študijskega informacijskega sistema, kamor ga je vnesel mentor. V nekaj stavkih bo opisal, kaj pričakuje od kandidatovega diplomskega dela. Kaj so cilji, kakšne metode uporabiti, morda bo zapisal tudi ključno literaturo.
\vspace{15mm}

\vspace{2cm}

% prazna stran
\clearemptydoublepage

% zahvala
\thispagestyle{empty}\mbox{}\vfill\null\it%
\noindent
Na tem mestu zapišite, komu se zahvaljujete za izdelavo diplomske naloge. Pazite, da ne boste koga pozabili. Utegnil vam bo zameriti. Temu se da izogniti tako, da celotno zahvalo izpustite.
\rm\normalfont

% prazna stran
\clearemptydoublepage

%%%%%%%%%%%%%%%%%%%%%%%%%%%%%%%%%%%%%%%%
% kazalo
\pagestyle{empty}
\def\thepage{}% preprecimo tezave s stevilkami strani v kazalu
\tableofcontents{}

% prazna stran
\clearemptydoublepage

%%%%%%%%%%%%%%%%%%%%%%%%%%%%%%%%%%%%%%%%
% seznam kratic

\chapter*{Seznam uporabljenih kratic}  % spremenil Solina, da predolge vrstice ne gredo preko desnega roba

\noindent\begin{tabular}{p{0.1\textwidth}|p{.4\textwidth}|p{.4\textwidth}}    % po potrebi razširi prvo kolono tabele na račun drugih dveh!
  {\bf kratica} & {\bf angleško}                             & {\bf slovensko} \\ \hline
  {\bf } & \\
  {\bf } & \\
  {\bf } & \\
  {\bf } & \\
%  \dots & \dots & \dots \\
\end{tabular}


% prazna stran
\clearemptydoublepage

%%%%%%%%%%%%%%%%%%%%%%%%%%%%%%%%%%%%%%%%
% povzetek
\addcontentsline{toc}{chapter}{Povzetek}
\chapter*{Povzetek}

\noindent\textbf{Naslov:} \ttitle
\bigskip

\noindent\textbf{Avtor:} \tauthor
\bigskip

%\noindent\textbf{Povzetek:} 
\noindent 
Za diplomsko delo sem se odločila razviti mobilno aplikacijo Asana Master, ki bi v neki celoti zaobjela potovanje posameznika po njegovi poti Joge. Asane oz. položaji imajo v Jogi velik pomen, zato je pomembno, da je izvajanje le teh pravilno in v skladu z zmožnostmi posameznika. Prav z namenom učenja pravilne izvedbe položajev, spremljanjem napredka in sledenja lastnim občutkom praktikantov, pa sem razvila aplikacijo, ki je primerna tako za začetne kot nadaljevalne praktikante, saj so Asane razdeljene na različne stopnje, prav tako pa si praktikant lahko sam določi svoje cilje. 

\bigskip

\noindent\textbf{Ključne besede:} \tkeywords.
% prazna stran
\clearemptydoublepage

%%%%%%%%%%%%%%%%%%%%%%%%%%%%%%%%%%%%%%%%
% abstract
\selectlanguage{english}
\addcontentsline{toc}{chapter}{Abstract}
\chapter*{Abstract}

\noindent\textbf{Title:} \ttitleEn
\bigskip

\noindent\textbf{Author:} \tauthor
\bigskip

%\noindent\textbf{Abstract:} 
\noindent 
For my dissertation, I decided to develop the Asana Master mobile app, which would completely encompass an individual's journey along his Yoga path. Asanas or positions are of great part in Yoga, so it is important that the implementation of these is correct and in accordance with the abilities of the individual. In order to learn how to perform the positions correctly, monitor progress and follow the practitioners' own feelings, I developed an application that is suitable for both beginners and advanced practitioners, as Asanas are divided into different levels, and the practitioner can set his own goals.
\bigskip

\noindent\textbf{Keywords:} \tkeywordsEn.
\selectlanguage{slovene}
% prazna stran
\clearemptydoublepage

%%%%%%%%%%%%%%%%%%%%%%%%%%%%%%%%%%%%%%%%
\mainmatter
\setcounter{page}{1}
\pagestyle{fancy}

\chapter{Uvod}
Joga je za nekoga lahko le oblika sprostitve, način napajanja z energijo, vadba za boljše počutje, za tiste, ki se z jogo ukvarjajo bolj intenzivno, pa je joga predvsem izjemno, do potankosti izdelano in premišljeno urjenje osebnosti, ki lahko človeku omogoči, da dozori. Joga je pot k sebi. Nima pa joga le pozitivnih učinkov na um, temveč seveda tudi na telo. 
Z leti postane naše telo manj gibčno, utrujeno, bolj občutljivo in nagnjeno k poškodbam, kar pa lahko učinkovito odpravimo z vadbo joge, saj telo postane prožnejše, bolj gibljivo in močnejše. Joga so telesni položaji ali asane preko katerih začutimo in spoznavamo svoje telo, ga nadzorujemo in izboljšujemo, vendar praksa asan ne sme biti brezglava in pretirana, zato je poznavanje joga položajev in pravilne izvedbe le-teh zelo pomembno. 
In kaj je asana? Sanskrtski izraz asana pomeni poza ali položaj telesa. Asana je psiho-somatska jogiska vaja za telo in um, saj z njo vplivamo tako na počutje telesa, kot tudi uma. Pozorni moramo biti tako na pravilen položaj telesa, kot tudi na naše dihanje. 

Prav z namenom pravilnega izvajanja položajev in spremljanjem napredka pri tem, pa sem se za svojo diplomsko nalogo odločila razviti mobilno aplikacijo, ki bi bodočim joga učiteljem in tudi vsem ostalim praktikantom joge omogočala lažje in bolj zabavno obliko učenja, ter dodatno motivacijo za napredek. Aplikacijo sem razvila z uporabo ogrodja React Native in odprtokodno platformo Expo. Za pisanje kode sem uporabila jezike, Java Script, TypeScript in SQL. Aplikacija teče na Expotovem lokalnem strežniku, za povezavo do baze podatkov pa je uporabljen strežnik Express. Podatki so shranjeni v MySql bazi, ki teče preko lokalnega MySQL strežnika.

\chapter{Joga}
\label{ch0}

\section{Zgodovina joge}

\textbf{Kaj je joga in njeno bistvo?}\\

Joge ne sestavljajo le telesni položaji ali asane. Prakticiranje joge sestavljajo tako fizične kot duhovne prakse, ki se med seboj razlikujejo glede na temperament. Praktikant, torej ta, ki se odloči za izvajanje duhovne prakse, si izbere tisto pot in tisto duhovno prakso, ki je njemu najbližja. Katera je to, lahko ugotovi le s preizkušanjem. Pri tem velja poudariti, da ni nobena boljša ali slabša. Vsekakor pa nekatere lahko določenemu značaju ustrezajo bolj, druge manj. Ko praktikant izbere neko pot, je potrebno, da se te poti drži redno, zavedno, predano in daljše časovno obdobje, saj le na ta način lahko izkusi sadove posameznih duhovnih tehnik in poglobi svoje zavedanje in smisel Joge. Cilj vseh tehnik je duhovni razvoj, razsvetljenje oziroma globoko spoznanje življenjskih resnic. \\ 

Kot posebna metoda preseganja človeške pogojenosti, je omenjena že v starodavnih vedskih spisih katerih nastanke zgodovinarji umeščajo na območje današnja Indije. Preden se je namreč razširila kot uveljavljena praksa so kot prevladujoči načini samo-preseganja služili predvsem obredni rituali žrtvovanja. Preko ritualov žrtvovanja, ki so vsebovali elemente smrti in ponovnega rojstva, so si ljudje poskušali zagotoviti moč nadzora nad lastno minljivostjo. 

V obdobju prvih upanišadskih spisov so obredni rituali začeli izgubljati svoj vlogo prevladujočega načina samo-preseganja. Zmožnost odrešitve je prenehala obstajati kot poseben privilegij pogojen s poznavanjem ritualnih praks, saj se je iz rituala, kot sredstva odrešitve, pozornost preusmerila na osebo in način njenega bivanja. Modreci ršiji - avtorji upanišadskih spisov - so namreč v svojih nazorih poudarjali, da je odrešitev zmožen doseči prav vsak z načinom svojega bivanja in delovanja.

To bivanje in delovanje so opisali v filozofskih nazorih samkhye iz katerih izhaja tudi filozofija joge. Filozofija samkhye temelji na predpostavki, da je človeško trpljenje posledica nepoznavanja samega sebe in vzrokov svojih dejanj. Sprememba načina bivanja, ki bi onemogočila trpljenje je pogojena spoznanjem in udejanjanjem samega sebe. To spoznanje ni razumsko spoznanje, saj spoznavajoči ne ustvarja spoznanja, ki bi bilo ločeno od njega, ampak je spoznanje oziroma se (sam sebi) razodeva kot spoznanje.

Spoznanje samega sebe se razodeva na nivoju skupne zavesti. Tu se jaz zaveda samega sebe neodvisno od fizičnega in psihičnega doživljanja. S tem, ko je zmožen kot pričujoča zavest bivati onstran fizičnega in psihičnega izkustva, lahko spoznava in spreminja pogoje lastnega bivanja in vzroke svojega delovanja.

Filozofija joge je prevzela osnovne predpostavke samkhye, le da odrešitve ne povezuje toliko s spoznanjem, kolikor z načinom delovanja. Človek je zmožen preseči lastno pogojenost in bivati neodvisno od materialnega sveta (za katerega se je predpostavljajo, da vsebuje tako fizično kot psihično realnost) prav z določenim načinom doživljanja in delovanja v materialnem svetu.

Omenjen način doživljanja in delovanja je v filozofiji joge opisan kot jogijska praksa. Prvo sistematično zbirko jogijskih praks, ki so se izvajale na območju današnje Indije je izdelal Patanjali v svojih Yoga-sutrah (njegovo življenje zgodovinarji umeščajo v obdobje od 2. stol. pr. n. št - 5. st. n. št.). V njih izpostavlja, kako praksa joge ne omogoča zgolj duhovne neodvisnosti od pogojev materialnega sveta, ampak tudi zmožnost njihovega spreminjanja. S pomočjo prakse joge, ki vključuje tako fizično, psihično kot duhovno komponento, smo zmožni spreminjati svojo nezavedno pogojenost in iz nje izhajajočo zmožnost delovanja.\\


Tradicionalno jogo sestavlja osem temeljnih vej (ash – osem, anga – veja), ki jih je prvič sistematično opisal Patanjali v svojih Yoga-sutrah:

\begin{enumerate}
	\item \textbf{Yama ali etične zmožnosti} 
		\begin{itemize}
			\item nenasilje (ahimsa) – nezmožnost škodovati sebi in drugim
			\item resnicoljubnost (satya) – usklajenost govora z dejanji
			\item onemogočanje kraje oziroma težnje po prilaščanju tega kar nam ne pripada (asteya)
			\item neodvisnost od spolnih potreb (brahmacarya)
			\item onemogočenje pohlepa (aparigraha) oziroma težnje po prilaščanju tega česar ne potrebujemo
		\end{itemize}
	
	\item \textbf{Niyama ali discipline}
		\begin{itemize}
			\item čiščenje telesa in duha (sauca)
			\item zmožnost izkusiti zadovoljstvo neodvisno od zunanjih dejavnikov (samtosa)
			\item askeza (tapas)
			\item odsotnost besed (kastha mauna)
			\item odsotnost gest in kretenj (akara mauna)
			\item študij načinov samo-preseganja
		\end{itemize}

	\item \textbf{Asana ali telesni položaji}
	\item \textbf{Pranayama ali dihalne tehnike}
	\item \textbf{Pratyahara ali neodvisnost čutil od zaznavanja zunanjih dejavnikov}
	\item \textbf{Dharana ali zbranost in zmožnost nadzorovanega usmerjanja pozornosti}
	\item \textbf{Dhyana ali meditacija}
	\item \textbf{Samadhi ali stanje skupne zavesti}
\end{enumerate}


\section{Asane in njihov pomen}
Beseda asana pomeni stabilen in udoben položaj. Največkrat pa jo v slovenščini enačimo s terminom vaja ali položaj.\\ 

V hatha jogi so asane ena osnovnih tehnik telesnega gibanja. Asane dajejo telesu gibljivost in pravilno držo, ki omogoča optimalno energetsko in telesno stabilnost, usmerjene so v živčni sistem, predvsem v hrbtenico, ki je po jogijski anatomiji osnovna energetska os našega bivanja.

Začetniki v hatha jogi so se morali najprej naučiti, kako pravilno stati, sedeti in ležati. Zato se jogijska praksa tudi bistveno razlikuje od vseh drugih vadb, saj zahteva poleg natančnega, pravilnega izvajanja asan tudi pozornost, usmerjenost uma v to, kar počnemo. V asani smo, kot pove že sanskrtsko ime "as" oz. biti, tukaj in zdaj.

Asan je nešteto, nekateri viri navajajo številko 70.–80.000. Poimenovali so jih po živalih, rastlinah, mineralih in  božanstvih. Bilo naj bi jih toliko, kolikor je oblik življenja na zemlji. Večina asan vsebuje hatha princip, ki pomeni princip moči. Zelo pomembno je, da asane izvajamo primerno svojim telesnim sposobnostim in pravilno, kar pomeni postopnost učenja in poslušanje svojega telesa. Nikoli ne izvajamo asan preko svojih zmožnosti, ampak le do praga bolečine. \\ 

Izvedba asane naj bi bila stabilna in hkrati selektivno sproščena. Dihanje je sproščeno, enakomerno, pozornost je usmerjena v počutje telesa, opazujemo sporočila telesa, umirimo um in zadržujemo položaj toliko časa, dokler čutimo asano v vseh navedenih lastnostih. 

Skozi asane se nam razkrivajo naši ustaljeni psihofizični vzorci, ki jih običajno doživimo kot omejitve. V jogi ni tekmovalnosti, vsak je v svoji asani, ki trenutno ustreza njegovemu psihofizičnemu stanju. Učitelji joge pogosto pravijo, da je tista asana, ki nam je najmanj všeč, pravzaprav naša asana.

Z rednim in discipliniranim izvajanjem asan se prožnost telesa izredno poveča. Mnoge asane učinkujejo tudi terapevtsko in lahko odpravijo težave, ki nas pestijo dolgo časa in so morda videti nerešljive, odpravi pa jih lahko že pravilna drža telesa. Hkrati se poveča tudi naša prožnost duha: opazimo lahko spremembe v svojem dojemanju sveta, samozavesti, odprtosti, sposobnosti, da ocenimo položaj in ugotovimo, kaj je bistvo in kaj nepomembna navlaka. Trdno stojimo na tleh, z odprtim srcem in notranjo radostjo, ki ne potrebuje zunanjih dražljajev. Takrat smo prestopili prvo, pomembno stopnico na poti joge. \\


»Bistvo izvajanja asane ni nujno v popolni izvedbi. Bistvo asane je v popolnem zavedanju.«


\chapter{Opis aplikacije}
\label{ch1}

\section{Ideja in zasnova aplikacije}

\section{Načrtovanje aplikacije in zbiranje podatkov}

\section{Glavne funkcionalnosti in delovanje aplikacije}

\chapter{Tehnologije in orodja}
\label{ch2}

\section{Zaledje aplikacije}


\chapter{Testno okolje}
\label{ch3}


\chapter{Prototip aplikacije}
\label{ch6}


\chapter{Zaključek}
\label{stroka}


\clearpage
\addcontentsline{toc}{chapter}{Literatura}
\bibliographystyle{plain}
\bibliography{literatura}

\end{document}

